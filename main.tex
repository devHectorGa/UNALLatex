\documentclass{article}
\usepackage[utf8]{inputenc}
\usepackage{amsmath}

\title{Matemática}
\author{Hector Ferney Garzón Cagua }
\date{Septiembre 2020}

\begin{document}

\maketitle
\tableofcontents

\section{Introducción}
Esta es un archivo para mostrar como funciona \LaTeX.

\section{Formas de escribir en latex}

\tiny
$ax+by+c=0$
$$ax+by+c=0$$
\( ax+by+c=0 \)

\begin{math}
  ax+by+c=0
\end{math}
\section{Fracción}

\begin{math}
  \frac{a}{b}=\frac{c}{d} \\
\end{math}

Tenemos la equivalencia $\frac{a}{b}=\frac{c}{d}$, válida para todo $a$, $b$, $c$, $d$ \\

\normalsize


Tenemos la equivalencia $$\frac{a}{b}=\frac{c}{d}$$ válida para todo $a$, $b$, $c$, $d$ \\

\begin{align*}
  x &=y    &   w &=z  &   a &= b+c \\
  2x &=-y  &      &   &    a&= b \\
\end{align*}

\begin{equation}
  \begin{matrix}
    a & b \\
    c & d
  \end{matrix}  
\end{equation}

\begin{equation}
  \begin{pmatrix}
    2 & 5 & 0 \\
    7 & 3 & 8 \\ 
    3 & 0 & 1
  \end{pmatrix}
\end{equation}

\begin{equation}
  \begin{bmatrix}
    2 & 5 & 0 \\
    7 & 3 & 8 \\ 
    3 & 0 & 1
  \end{bmatrix}
\end{equation}

\begin{equation}
  \begin{Bmatrix}
    2 & 5 & 0 \\
    7 & 3 & 8 \\ 
    3 & 0 & 1
  \end{Bmatrix}
\end{equation}

\[
  \sum_{i=1}^{n} x_i = x_{1}+x_{2}+x_{3}
\]

\[
  \sum_{i=1}^{n} \frac{\cos(i)}{\arccos{2i}}
\]

\end{document}
